% !TEX program = xelatex
\documentclass{resume}
\usepackage{setspace}
%\usepackage{zh_CN-Adobefonts_external} % Simplified Chinese Support using external fonts (./fonts/zh_CN-Adobe/)
%\usepackage{zh_CN-Adobefonts_internal} % Simplified Chinese Support using system fonts

\begin{document}

\pagenumbering{gobble} % suppress displaying page number

\name{Hongliang He}

\basicInfo{
  \email{3160103176@zju.edu.cn} \textperiodcentered\ 
  \phone{(+86) 188-6810-8570} \textperiodcentered\ 
  }

\section{\faGraduationCap\ Education}
\datedsubsection{\textbf{Zhejiang University (ZJU), Hangzhou, China}}{Sept. 2016 -- Present} 
Major in Interdisciplinary Innovation Platform (Computer Science and Technology \& Control) 

\textbf{Chu Kochen Honors College} \hfill Overall GPA: 3.67/4.0\\
Core Courses:\\
Computer Vision (5.0/5.0) \hfill Fundamentals of Programming (4.8/5.0)\\
Probability Theory and Mathematical Statistics (4.8/5.0) \hfill Data Analysis and System Identification (4.8/5.0)\\
Technology and Strengthening Practice of Wheeled Robot (4.8/5.0)\hfill Artificial Intelligence (4.5/5.0)

\section{\faUsers\ Experience}
\datedsubsection{\textbf{Return Points and Battery Replacement Strategy for E-BIKE }}{ Mar. 2018 -- May 2018}
\role{MATLAB, Windows}{Mathematical modeling, Group Leader \& Programmer}
Responsibility: 
\begin{itemize}
  \item Implemented the gradient descent algorithm and found the optimal return point in the grid map;
  \item Implemented a model and calculate the battery replacement frequency.
\end{itemize}

\datedsubsection{\textbf{Mobile Robot Navigation and Obstacles Avoidance}}{ Sept. 2018 -- Jun. 2019}
\role{Python, C++, Linux}{Individual Project, collaborated with a classmate}
Responsibility: 
\begin{itemize}
  \item Early-work: traditional heuristic search algorithms (A*, RRT...) were implemented on small soccer robots;
  \item Mid-term: Implementing reinforcement learning algorithms (value-based methods and policy gradients);
  \item Post-work: Compared traditional algorithms with RL algorithms and formed an English paper.
\end{itemize}

\datedsubsection{\textbf{Chinese Chatbot based on Seq2Seq and Knowledge}}{Mar. 2019 -- Jun. 2019}
\role{Python, MySQL, Windows}{Individual Project, collaborated with a classmate}
Responsibility: 
\begin{itemize}
  \item Implemented Seq2Seq model with Pytorch;
  \item The way to get answers from database according to the input questions;
  \item A Python-Spider used to get corpus. 
\end{itemize}

\datedsubsection{\textbf{MAML with Natural Gradient Adaptation}}{Jun. 2019 -- Present}
\role{Python}{Scientific Research Training, idea from Dr. Zebang Shen}
\textit{Supervised by Prof. Hui Qian, Artificial Intelligence Laboratory, Zhejiang University}\\
Responsibility: 
\begin{itemize}
  \item Learning Reinforcement Learning concepts and algorithms.
  \item I'm now doing experiments (tensorflow) of \textit{ Model-Agnostic Meta-Learning with Natural Gradient Adaption}.
\end{itemize}

% Reference Test
%\datedsubsection{\textbf{Paper Title\cite{zaharia2012resilient}}}{May. 2015}
%An xxx optimized for xxx\cite{verma2015large}
%\begin{itemize}
%  \item main contribution
%\end{itemize}

\section{\faCogs\ Skills}
\begin{itemize}[parsep=0.5ex]
  \item Programming Languages: C/C++, Python, MATLAB
\end{itemize}

\section{\faHeartO\ Honors and Awards}
\datedline{\textit{Academic Third-Class Scholarship} Award by Zhejiang University }{2017 \& 2018}
\datedline{\textit{The Second Place of Taekwondo Competition 72kg} Award by Zhejiang University}{2017}
\datedline{\textit{Rank 13(top 7\%) in Zhejiang University ACTF Cyber Security Competition}}{2019}
\section{\faInfo\ Miscellaneous}
\begin{itemize}[parsep=0.5ex]
  \item GitHub: https://github.com/MinorJerry
\end{itemize}

%% Reference
%\newpage
%\bibliographystyle{IEEETran}
%\bibliography{mycite}
\end{document}
